\documentclass[12pt, a4paper]{article}

\usepackage[margin=1in]{geometry}
\usepackage{amsmath}
\usepackage{amssymb}
%\usepackage{sourcecodepro}
\usepackage{courier}
\usepackage{listings}
\usepackage{lstautogobble}
\usepackage[dvipsnames]{xcolor}

\setlength{\parskip}{\baselineskip}
\setlength{\parindent}{0pt}

\definecolor{comment}{HTML}{3D6768}

\lstset{language=Octave,
  basicstyle=\footnotesize\ttfamily,
  commentstyle=\textcolor{comment},
  breaklines=true,
  autogobble=true}

\renewcommand*\rmdefault{bch}

\begin{document}

\begin{titlepage}
  \centering
  \vspace*{10cm}
  {\huge\bfseries Machine Learning} 
  
  {\large\bfseries Coursera Notes}

  {12/12/2016 - 06/03/2017}
\end{titlepage}

  \newpage

\tableofcontents

  \pagenumbering{gobble}

  \newpage

  \pagenumbering{arabic}
  
\section{Introduction} 

  Any machine learning problem can be assigned one of two broad 
  classifications:

  \textbf{Supervised Learning}
  \begin{itemize}
    \item The algorithm predicts the right answer (i.e. the output) 
      based on the assumption of there being a relationship in the 
      training set between the input and the output.
    \item Regression problem: predict continuous-valued output.
    \item Classification problem: predict discrete-valued output.
    \item Possible to include infinite number of variables.
  \end{itemize}

  \textbf{Unsupervised Learning}
  \begin{itemize}
    \item No output: tasks the algorithm with finding structure in 
      data.
    \item Algorithm can cluster/separate data.
  \end{itemize}

  \newpage
	

\section{Univariate Linear Regression}
 
  \textbf{Notation:}
  \vspace{-1em}
  \begin{itemize}
    \item Training set: $(x^i, y^i)$.
    \item $m$ = number of training examples.
    \item The learning algorithm takes a training set and outputs $h$.
    \item Hypothesis function $\hat{y}$ represented as $h_\theta(x) = 
          \theta_0 + \theta_1x$,
    \item $\theta_i$: parameters
    \item $\theta_0$: $y$-intersect
    \item $\theta_1x$: gradient
    \item $\alpha$: learning rate (greater = more aggressive)
  \end{itemize}

  \subsection{Cost Function}

    The squared error cost function measures the accuracy of the 
    hypothesis using the average difference of each hypothesis taken
    from $(x_i, y_i)$:

    $J(\theta_0, \theta_1) = \dfrac {1}{2m} \displaystyle 
    \sum_{i=1}^m \left ( \hat{y_{i}}- y_{i} \right)^2 = \dfrac 
    {1}{2m} \displaystyle \sum_{i=1}^m \left (h_\theta (x_{i}) 
    - y_{i} \right)^2$

    The mean of the squares is halved for convencience of computing the 
    gradient descent, as the derivative term of the square function will
    cancil out the 1/2 term.

    The goal is to minimise $J(\theta_0, \theta_1)$. The lowest 
    value for $J(\theta_1)$ is selected, which will return the 
    least difference (best fit).

  \subsection{Paramater Learning}

    Gradient Descent: find the quickest path to the local optimum. 
    If univariate, there will be just one global optimum. This is used to
    estimate the parameters in the hypothesis function.

    Repeat until convergence:

    $\theta_j := \theta_j - \alpha\frac {\partial}
    {\partial\theta_j} J(\theta_0, \theta_1)$

    \newpage

    Translates to:

    $\theta_0 := \theta_0 - \alpha \frac{1}{m} 
    \sum\limits_{i=1}^{m}(h_\theta(x_{i}) - y_{i})$

    $\theta_1 := \theta_1 - \alpha \frac{1}{m} 
    \sum\limits_{i=1}^{m}((h_\theta(x_{i}) - y_{i})x_i)$

    The derivative refers to the direction taken by the gradient descent 
    (i.e. positive or negative slope direction).

    Need to simultaneously update $\theta_0$, $\theta_1$..., 
    $\theta_n$ by assigning the computed gradiant descents for 
    $j$=0,$j$=1...,$j$=n to each variable at the same time. If $\theta_0$ is 
    updated before $\theta_1$, the second computation will be using the newly 
    assigned value of $\theta_0$.

    With fixed $\alpha$, the derivative will naturally decrease, 
    taking smaller steps to reach the convergence. $\alpha$ can take 
    too many steps to converge if too low, or alternatively can 
    overshoot and diverge if too high. If already at convergence, 
    the derivative will be treated as 0 and no more calculations 
    will be made. 

  \newpage

\section{Linear Algebra (Matrices and Vectors)}

  \textbf{Matrix}: 2-dimensional array of numbers.

  Dimensions: rows x columns, $\mathbb{R}$ 3x2

  $A_{ij}$ denotes the entry in the $i^{th}$ row, $j^{th}$ column.\\

  \textbf{Vector}: An $n$ x 1 matrix.

  A 4-dimensional vector contains 4 columns.

  $y_i$ refers to the $i^{th}$ element (could be 0 or 1-indexed).

  \subsection{Operations}

    To add/subtact two matrices, they must be the same dimension.
    
    $\begin{bmatrix} a & b \\ c & d \end{bmatrix} + \begin{bmatrix} w & x \\ 
    y & z \end{bmatrix} = \begin{bmatrix} a+w & b+x \\ c+y & d+z \end{bmatrix}$
    
    Scalar multiplication:

    $\begin{bmatrix} a & b \\ c & d \end{bmatrix} * x = \begin{bmatrix} a*x & 
    b*x \\ c*x & d*x \end{bmatrix}$

    Matrix-vector multiplication: a \textbf{m x n matrix} multiplied by a  
    \textbf{n x 1 vector} becomes a \textbf{m x 1 vector}. 

    $\begin{bmatrix} a & b \\ c & d \\ e & f \end{bmatrix} * \begin{bmatrix} 
    x \\ y \end{bmatrix} = \begin{bmatrix} a*x+b*y \\ c*x+d*y \\ e*x+f*y
    \end{bmatrix}$

    Matrix-matrix multiplication: A \textbf{m x n matrix} mutliplied by a 
    \textbf{n x o matrix} results in a \textbf{m x o matrix}.

    $\begin{bmatrix} a & b \\ c & d \\ e & f \end{bmatrix} * \begin{bmatrix} 
    w & x \\ y & z \end{bmatrix} = \begin{bmatrix} a*w+b*y & a*x+b*z \\ 
    c*w+d*y & c*x+d*z \\ e*w+f*y & e*x+f*z \end{bmatrix}$

  \newpage

  \subsection{Properties}

    \begin{itemize}
      \item Not commutative, A x B != B x A.
      \item Is associative, A x B x C calculated as A x (B x C) and
        (A x B) x C.
      \item Identity Matrix (1 is the identity, denoted $I$). 
        Is commutative!
    \end{itemize}		    

  \subsection{Inverse and Transpose}

    \textbf{Matrix Inverse}:

    If A is an m x m matrix, and has an inverse,
    $A(A^{-1}) = A^{-1} A = I$.		  

    Only square matrices can have an inverse, i.e. 2x2 x inverse = 
    $I_{\text{2x2}}$

    $I_{\text{2x2}}$ being $\begin{bmatrix} 1 & 0 \\ 0 & 1 \end{bmatrix}$

    \textbf{Matrix Transpose}:

    $B_{ij} = A_{ji}$

    e.g. 
    $A = \begin{bmatrix} 1 & 2 & 0 \\ 3 & 5 & 9 \end{bmatrix}$
    $A^T = \begin{bmatrix} 1 & 3 \\ 2 & 5 \\ 0 & 9 \end{bmatrix}$	    

  \newpage

\section{Multivariate Linear Regression}

  $x_1,... x_n$ denote variables.\\
  $x^{(i)}$: features of $i^{th}$ training example, shown as vector 
      (e.g. the $i^{th}$ row).\\
  $x_j^{i}$: value of feature $j$ in $i^{th}$ training example.
  $n$: number of features $x$.\\
  $m$: training examples.\\
  $y$: ouptut variable.\\

  $h_\theta^{(x)} = \theta_0 + \theta_1x_1 + \theta_2x_2 + ... + 
    \theta_nx_n$

  $x = \begin{bmatrix} x_0 \\ x_1 \\ ... \\ x_n \end{bmatrix} ER^{n+1}$,
    $\theta = \begin{bmatrix} \theta_0 \\ \theta_1 \\ ... \\ \theta_n 
    \end{bmatrix} ER^{n+1}$

  For convenience, let $x_0$ = 1. So $h_\theta^{(x)} = \theta_0x_0 + ...
    \theta_nx_n = \theta^Tx$

  $\theta^T$ = (n+1) x 1 matrix 

  We can think about $\theta_0$ as the basic price of a house, $\theta_1$ 
    as the price per square meter, $\theta_2$ as the price per floor, etc. 
    $x_1$ will be the number of square meters in the house, $x^2$ the 
    number of floors, etc. (and $x_0$ is 1).

  \subsection{Gradient Descent}

    Parameters: $\theta$, an $n$+1-dimensional vector.

    Repeat until convergence: 
    $\theta_j := \theta_j - \alpha \frac{1}{m} \sum\limits_{i=1}^{m} 
      (h_\theta(x^{(i)}) - y^{(i)}) \cdot x_j^{(i)} \text{for j := 0...n}$
    
    Simultaneous update applies.

    Feature scaling: variables should be similarly scaled by dividing $x_i$
      by the range (or standard deviation), else contours will skew, making 
      it harder to reach global minimum. Get every feature into an 
      approximate $-1 \leq x_i \leq 1$ range. 
    
    Mean normalisation: $x_i := \frac{x_i - \mu_i}{s_1}$, where $\mu$ is
      the mean and $s_1$ the range or standard deviation.

    Look at a convergence test plot to see if gradient descent is working
      correctly. Change the $\alpha$ until the global minimum is reached. 
      $J(\theta)$ should be decreasing on every iteration. Try 3-fold or 
      10-fold increases in $\alpha$ for instance. 

  \subsection{Features and Polynomial Regression}

    By combining features, e.g. $x_1$ is frontage and $x_2$ is depth, which
    can be multiplied to create a new feature, $x$ (area), the hypothesis 
    may be easier to calculate. 

    \textbf{Polynomial Regression:} For a potentially better fit use:
    \begin{itemize}
      \item \emph{quadratic function} $h_\theta(x) = \theta_0 + \theta_1x + 
        \theta_2x^2$ 
      \item \emph{cubic function} $h_\theta(x) = \theta_0 + \theta_1x + 
        \theta_2x^2 + \theta_3x^3$
      \item \emph{square root function} $h_\theta(x) = \theta_0 + 
        \theta_1 x_1 + \theta_2 \sqrt{x_1}$
    \end{itemize}
    where $x$ is a feature such as the size of a house.

  \subsection{Normal Equation}

  Normal equation is a method to solve for $\theta$ analytically.

  Take $x_i$ training sets and place them in an $X$ matrix. Add the $y$ 
    training set to a $y$ matrix. The $X$ matrix should be m x (n+1) and the
    vector m-dimensional. 

  i.e. design matrix $X = \begin{bmatrix} (x^{(1)})^T \\ \vdots \\ 
    (x^{(m)})^T \end{bmatrix}$

  Remember that $x_0$ = 1.
  
  $\theta = (X^TX)^{-1}X^Ty$

  $(X^TX)^{-1} \text{is the inverse of matrix} X^TX$.

  In Octave: \texttt{pinv(X'*X)*X'*y}

  \textbf{Normal Equation Rules:}
  \begin{enumerate}
    \item No need to use feature scaling.
    \item No need to choose $\alpha$.
    \item No need to iterate.
  \end{enumerate}

  \newpage

  \textbf{However:}
  \begin{itemize}
    \item Need to compute $X^TX)^{(-1)}$
    \item Slow if $n$ is very large (consider using gradient descent if
      $n >$ 10,000).
  \end{itemize}

  \textbf{Noninvertibility:}

  Not every matrix $X^TX$ is invertible. These are called noninvertible/\\
  singular/degererate matrices.

  Noninvertibility occurs when matrices contain:
  \begin{itemize}
    \item Redundant features (linearly dependent i.e. different units for 
          the same feature, such as size in metres and square feet).
    \item Too many features (i.e. if $m \leq n$).
  \end{itemize}

  \newpage

\section{Octave}

  \textbf{Misc operators and functions:}
  \begin{lstlisting}
    ~=                      % Not True
    &&                      % AND
    ||                      % OR (Can also use xor(n, m))
 

    disp()                  % print function
    sprintf()               % make string (sep=', quote='')
    format long, short      % show floating point decimals

    % semicolon suppresses output, comma chains commands

  \end{lstlisting}

  \subsection{Vectors and Matrices:}

    \begin{lstlisting}
    A = [1 2; 3 4; 5 6]     % return a 3x2 matrix
    B = [1; 2; 3]           % return a 1x3 matrix
    V = [1 2 3]             % return a 3-Dimensional vector

    V = 1:0.1:2             % return a vector starting at 1, ending
                            % at 2, increments of 0.1 inbetween

    ones(2,3)               % return a 2x3 matrix containing all 1
                            % multiply by 2 for all 2
    zeros(n, m)             % all 0

    rand(n, m)              % all random between 0-1
    randn(n, m)             % all random from Guassian distribution

    eye(n)                  % return an n x n identity matrix

    hist(A, n)              % plot a histogram for matrix with 
                            % n bins   
 
    VAR(n:m)                % show vector elements n to m
                            % show matrix element (n, m)
                            % (n,:) to show all in n row
                            % (:,n) to show all in n column
                            % use ([n m]) to show multiple
  
    A(:,n) = [x, y, z]      % assignment
    A = [A, [x, y, z]]
        or  [x; y; z]

    C = [A B]               % concatenate horizontally
        [A; B]              % vertically

    A(:)                    % put all matrix elements into vector

    size(A, n)              % return size of matrix in a matrix
                            % n=1 for rows, 2 for columns
    length(V)               % return size of vector
    \end{lstlisting}

  \subsection{Managing Data}

    \begin{lstlisting}
    load FILE               % load data in file to Octave

    who                     % show variables in current scope
    whos                    % detailed info about variables
    
    clear VAR               % remove variable (no arg=all)

    save FILE VAR           % save VAR to FILE (.txt or .mat)
                            % -ascii option at end if .txt

    addpath(`dir')          % assert Octave search path
    \end{lstlisting}
    \vspace{1em}

  \subsection{Computational Operations}

    \begin{lstlisting}
    A * B                   % multiply 
    A .* B                  % multiply corresponding elements
    A .^ 2                  % square each element
    1 ./ A                  % reciprical for each element

    log(A)                  % element-wise logorithm
    exp(A)                  % element-wise exponential 
    abs(A)                  % element-wise absolute value

    -A                      % negative elements
    A + 1                   % add 1 to each element

    A'                      % transpose

    max(V)                  % return max value
                            % returns max value row for matrix
    [val, ind] = max(V)     % assign max value and its index
    max(A,[],n)             % n=1 returns max column, 2 max row
                            % for matrices
    max(max(A))             % return max value of max row
    max(A(:))               % convert to vector to find abs max

    A < n                   % returns True/False for each element
    find(A < n)             % returns real value of True elements

    [x,y] = find(A > n)     % assign index to x and y variables
    
    magic(n)                % generate n x n matrix where each
                            % row/column adds up to same value

    sum(A, n)               % return sum of all elements
                            % n=1 returns column sum, 2 row sum
    prod(A)                 % return product (multiplication)
    floor(A)                % round down
    ceil(A)                 % round up

    pinv(A)                 % return inverse
    \end{lstlisting}
   
  \newpage

  \subsection{Plotting Data}

    \begin{lstlisting}
    plot(x, y)              % plot x against y
    hold on                 % continue using current plot
    xlabel(), ylabel()      % label x and y-axes `string'
    legend()                % provide legend
    title()                 % provide title

    print -dpng `name'      % save as png

    clf                     % clear figure
    close                   % close figure

    figure(n): plot()       % assign a plot to figure n
    
    subplot (n, m, x)       % divide n x m grid, access element x
                            % then use plot() to plot in element

    axis([x1 x2 y1 y2])     % set x and y ranges

    imagesc(A)              % plot matrix as coloured grid
    colorbar                % show colourbar legend
    colormap COLOUR         % change colour of map
    \end{lstlisting}
    \vspace{1em}

  \subsection{Control Statements}

    \begin{lstlisting}
    % for loops
    % n:m is the range
    for i=n:m,
      statement;
    end;

    % while loops
    i = n
    while true,
      statement;
      if i == n,
        break;      % break, continue available
      end;
    end;

    % if, elsif and else statements
    if v(i) == n,
      statement;
    elseif v(i) == m,
      statement;
    else
      statement;
    end;

    % functions
    % write in function_name.m file and cd into dir
    function y = function_name(x)        % args = x

    y = x + 1;

    % multiple outputs
    function [y1, y2] = function_name(x)

    y1 = x + 1;
    y2 = x + 2;

    % call in interpreter
    >> function_name(5)
    ans = 6

    % cost function
    function J = costFunctionJ(X, y, theta)   % X is design matrix
                                              % y is class labels

    m = size(X, 1);                           % training examples 
    predictions = X*theta;                    % predictions of h on all 
                                              % m examples
    sqrErrors = (predictions-y) .^ 2;         % squared errors

    J = 1/(2*m) * sum(sqrErrors);             % final cost function
    \end{lstlisting}
    \vspace{1em}

  \subsection{Vectorisation}

    Think of $h_\theta(x) \text{ as } \theta^Tx$.

    \begin{lstlisting}
    % unvectorised implementation
    prediction = 0.0;
    for j = 1:n+1,
      prediction = prediction + theta(j) * x(j)
    end;

    % vectorised implementation
    prediction = theta' * x;
    \end{lstlisting}
    
    For simultaneously updated gradient descent, $\theta_0, \dots \theta_n$
    (as seen on page 3) can be thought of as:

    $\theta := \theta - \alpha\delta$
    
    where $\delta = \frac{1}{m} \displaystyle \sum_{i=1}^m 
    (h_\theta(x^{(i)}) - y^{(i)}) x^{(i)}$

    $\theta = \mathbb{R}^{n+1} \newline \alpha = \mathbb{R}
    \newline \delta = \mathbb{R}^{n+1} 
    \newline \alpha\delta = \mathbb{R}^{n+1}$

    $(h_\theta(x^{(i)}) - y^{(i)}) = \mathbb{R} \newline
    x^{(i)} = \mathbb{R}^{n+1}$

    $u(j) = 2v(j) + 5w(j)$ (for all j)
    vectorises to: $u = 2v + 5w$.

  \newpage
  
  \subsection{Linear Regression}
  
    \begin{lstlisting}
    % remember to add 1s to the first column of the design matrix

    % compute cost function
    function J = computeCost(X, y, theta)

    m = length(y);    
    h = X * theta;
    errors = h - y;
    sqrErr = error .^ 2;

    J = (1/2*m)) * sum(sqrErr);

    % gradient descent
    function theta = gradientDescent(X, y, theta, alpha, iterations)

    m = length(y);
    for iter = 1:iterations
      h = X * theta;
      errors = h - y;
      theta_change = alpha * (1/m) * (X' * errors);

      theta = theta - theta_change;

    % as the above functions are vectorised they will also work for
    % multivariate linear regression

    % feature normalisation
    function [X_norm, mu, sigma] = featureNormalise(X)

    m = length(X);
    mu = mean(X);
    sigma = std(X);
    mu_matrix = ones(m, 1) .* mu;
    sigma_matrix = ones(m, 1) .* sigma;

    X_norm = (X - mu_matrix) ./ sigma_matrix;

    % prediction
    prediction = [1, x1, x2, ...] * theta  

    % see docs/ML/ex1/ for info on how to plot graphs
    
    % normal equation
    function [theta] = normalEqn(X, y)

    theta = pinv(X' * X) * X' * y;
    \end{lstlisting}

    \newpage

\section{Classification and Representation}

  $y \in\{0,1\}$ where 0 = negative, 1 = positive.

  Binary and multiclass classification problems exist. 

  Using linear regression is unreliable as outliers can skew hypothesis and
  $h_\theta(x)$ can scale beyond 0 and 1.

  \textbf{Logistic Regression}: $0 \leq h_\theta(x) \leq 1$

  Sigmoid function: $h_\theta(x) = g(\theta^Tx), \text{ where } 
  g(z) = \frac{1}{1+e^{-z}}$ \text{ and } $z: \mathbb{R}$

  So, $h_\theta^{(x)} = \frac{1}{1+e^{-\theta^Tx}}$

  Interpretation: $h_\theta(x)$ = 0.7 = 70\% chance of positive outcome.

  $h_\theta(x) = P(y=1|x;\theta)$ i.e. the probability that y=1, given x
  parameterised by $\theta$.

  $P(y=0|x;\theta) + P(y=1|x;\theta) = 1$

  Predicts $y=1$ when $h_\theta(x) \geq 0.5$, or when $\theta^Tx \geq 0$

  Decision boundaries can be linear and non-linear. They are defined
  by the chosen theta.

  \subsection{Cost Function}

  Using logistic regression cost function returns a non-convex function: not
  guaranteed to converge to global minimum.

  $J(\theta) = \frac{1}{m} \displaystyle\sum_{i=1}^{m} \text{Cost}
  (h_\theta(x^{(i)}),y^{(i)})$
  
  Cost$(h_\theta(x^{(i)}, y^{(i)}) = \begin{cases}
    -\log(h_\theta(x)) \text{ if } y=1\\
    -\log(1-h_\theta(x)) & \text{if } y=0 \end{cases}$

  If $y=1, h_\theta(x)\rightarrow0$ pays a higher cost.\\ 
  If $y=0, h_\theta(x)\rightarrow1$ pays a higher cost.

  Simplified: Cost$(h_\theta(x),y) = -y\log(h_\theta(x)) 
  - (1-y)\log(1-h_\theta(x))$

  If $y=1$, the second half of the cost function is ignored.\\
  If $y=0$, the first half of the cost function is ignored.
  
  This method is derived from maximum likelihood estimation.

  \subsection{Gradient Descent}

    $J(\theta) = -\frac{1}{m}[\displaystyle\sum_{i=1}^m 
    y^{(i)}\log h_\theta(x^{(i)}) + (1-y^{(i)})\log(1-h_\theta(x^{(i)}))]$

    $\min_\theta J(\theta):$

    $\text{Repeat} \{\\ \-\hspace{1cm} \theta_j := \theta_j - 
      \alpha\frac{\partial}{\partial\theta_j}J(\theta)\\
      \hspace{1em} \}  \hspace{1em}$ (simultaneously update all $\theta_j$) 

    $\frac{\partial}{\partial\theta_j}J(\theta) =
    \displaystyle\sum_{i=1}^m (h_\theta(x^{(i)}) - y^{(i)})x_j^{(i)}$

    The difference between linear and logistic regression gradient descent is 
    that the definition of $h_\theta^{(x)}$ is now equal to 
    $\frac{1}{1+e^{-\theta^Tx}}$.
  
  \subsection{Optimisation}

    Algorithms:
      \begin{itemize}
        \item Gradient descent
        \item \textbf{Conjugate gradient
        \item BFGS
        \item L-BFGS}
      \end{itemize}

    Advantages: automatically pick $\alpha$; faster than gradient descent.
    Disadvantages: more complex

    Octave has a built in library containing implementations of these algorithms.
    Be careful with third party libraries as they vary in optimisation. 

    \begin{lstlisting}
    % cost function J(theta)
    % 5 used as example expected theta
    function [jVal, gradient] = costFunction(theta)

    jVal = (theta(1) - 5) ^2 + (theta(2) - 5) ^2;     % compute J(theta)
    gradient = zeros(2,1);                            
    gradient(1) = 2 * (theta(1) - 5);                 % derivative for
    gradient(2) = 2 * (theta(2) - 5);                 % theta_0 and theta_1

    % optimisation
    options = optimset(`GradObj', `on', `MaxIter', `100');
    initialTheta = zeros(2,1);
    [optTheta, functionVal, exitFlag] = ...
            fminunc(@costFunction, initialTheta, options);
    \end{lstlisting}

  \subsection{Multi-class Classification}

    $y\in\{0,1,2,\dots,n\} \text{ or } y\in\{1,2,3,\dots,n\}$

    $n$ symbols used for plotting on graph.

    \textbf{One-vs-all} classification:

    Each $y$ is assigned a class, $n$, which is referred to as 
    $h_\theta^{(i)}(x)$.

    Logistic regression is run $n$ times, with the class $i$ a postive value and
    the remaining classes a negative value. 

    $h_\theta^{(i)} = P(y=i|x;\theta) \hspace{1em} (i = 1, 2,\dots,n)$

    To make a prediction, pick $\max_i(h_\theta^{(i)}(x)$

    \newpage

\section{Regularisation}

  Underfitting: High bias; doesn't fit data well enough.\\
  Overfitting: High variance: fits the data well but not with new examples.

  Applies to both linear and logistic regression.

  Addressing overfitting:
  \vspace{-1em}
  \begin{enumerate}
    \item Reduce number of features (manually or by using an algorithm).
    \item Regularisation
      \begin{itemize}
        \item Keep all features but reduce values of parameters $\theta_j$.
        \item Works well with lots of features, each contributing to 
          predicting $y$.
      \end{itemize}
  \end{enumerate}

  \subsection{Cost Function}

    By awarding small values for parameters $\theta_0,\theta_1,\dots,\theta_n$
    it's possible to get a simpler hypothesis which is less prone to 
    overfitting.

    e.g. if $x_3, x_4$ are very high, $\theta_3 \text{ and } \theta_4$ can
    be valued at close to 0. 

    If there's no way of knowing, the cost function can be adjusted using
    regularisation to reduce all $\theta^{(x)}$ values:

    $J(\theta) = \frac{1}{2m} [\displaystyle\sum_{i=1}^m 
    (h_\theta(x^{(i)}) - y^{(i)})^2 + \lambda \displaystyle\sum_{j=1}{n} 
    \theta_j^2]$

    $min_\theta J(\theta)$

    If $\lambda$ is set too high, $\theta^{(x)}$ values will be penalised to
    hard and $h_\theta(x)$ will be represented primarily by $\theta_0$. This
    will result in an underfitting of the data. 

  \subsection{Linear Regression}

    Separate the function $min_\theta J(\theta)$ into $\theta_0$ and $\theta_j$
    because $\theta_0$ is not being penalised.

    $\theta_j := \theta_j - \alpha [ \frac{1}{m} \displaystyle\sum_{i=1}^m
    (h_\theta(x^{(i)}) - y^{(i)})x_j^{(i)} + \frac{\lambda}{m}\theta_j]$

    While $\theta_0$ will not be regularised. 

    Can also be thought of as $\theta_j := \theta_j(1-\alpha\frac{\lambda}{m})
    - \alpha [ \frac{1}{m} \displaystyle\sum_{i=1}^m (h_\theta(x^{(i)}) - 
    y^{(i)})x_j^{(i)}$ 

    where $1 < \alpha\frac{\lambda}{m} < 1$

    \subsubsection{Normal Equation}

      Minimise cost function by using:

      $\theta = (X^TX + \lambda \begin{bmatrix} 0 & 0 & 0 \\  0 & 1 & 0 \\
      0 & 0 & 1 \end{bmatrix}) ^{-1} X^Ty$

      The matrix is (n+1)x(n+1), in this case n=2.

    \subsubsection{Non-invertibility}

      If $m \leq n$ the normal equation is non-invertible. Using \texttt{pinv}
      will be unreliable. 

      Regularisation using the equation above for normal equation also takes
      care of non-invertibility. 

  \subsection{Logistic Regression}

    Separate $\theta_0 \text{ and } \theta_j$ as with linear regression.

    $\theta_j := \theta_j - \alpha [\frac{1}{m} \displaystyle\sum_{i=1}^m
    (h_\theta(x^{(i)}) - y^{(i)}) x_j^{(i)} + \frac{\lambda}{m} \theta_j]$

    \subsubsection{Advanced Optimisation}

      \begin{lstlisting}
      % cost function passed to fminunc
      jVal =          % code to compute J_theta + regularisation
      gradient(1) =   % code to compute derivative J_theta
      gradient(2) =   % code to compute derivative J_theta + regularisation
      gradient(n+1) =  % code to compute derivative J_theta + regularisation
      \end{lstlisting}

  \newpage

\section{Neural Networks}

  \subsection{Representation}

    A neural network is an algorithm that can be used for learning non-linear
    hypotheses. They are preferred if there are lots of features. Using 
    logistic regression can take thousands/millions of polynomial features 
    to compute if there are many features. 

    Single neuron equivalent to a linear regression function. A network is
    comprised of a group of neurons and contains 3 layers: the input layer, 
    the hidden layer and the output layer.

    $a_i^{(j)}$ = activation of unit $i$ in layer $j$. 

    $\Theta^{(j)}$ = matrix of weights (parameters) controlling function
    mapping from layer $j$ to layer $j+1$.

    If network has $s_j$ units in layer $j$ and $s_{j+1}$ units in layer
    $s_{j+1}$, then $\Theta^{(j)}$ will be of dimension $s_{j+1} x (s_j + 1)$.

    Within the hidden layer, for 3 features there will be 3 activations:

    $a_1^{(2)} = g(\Theta_{10}^{(1)} x_0 + \Theta_{11}^{(1)} x_1 + 
    \Theta_{12}^{(1)} x_2 + \Theta_{13}^{(1)} x_3)$

    $a_2^{(2)} = \dots$

    $a_3^{(2)} = g(\Theta_{30}^{(1)} x_0 + \Theta_{31}^{(1)} x_1 + 
    \Theta_{32}^{(1)} x_2 + \Theta_{33}^{(1)} x_3)$

    The output layer then calculates: 

    $h_\Theta(x) = a_1^{(3)} = g(\Theta_{10}^{(2)}a_0^{(2)} + 
    \Theta_{11}^{(2)}a_1^{(2)} + \Theta_{12}^{(2)}a_2^{(2)} +
    \Theta_{13}^{(2)}a_3^{(2)}$ 

    \subsubsection{Forward propogation: Vectorised implementation}
    
      $a_1^{(2)} = g(z_1^{(2)})$
      
      $a_2^{(2)} = \dots$

      $a_3^{(2)} = g(z_3^{(2)})$

      \textbf{So}: $a^{(2)} = g(z^{(2)})$, where $z^{(2)} = \Theta^{(1)}x$.

      $a$ and $z$ are both $n$-dimensional vectors so the sigmoid is applied 
      element-wise as in logistic regression.

      The bias unit also needs to be added, which remains 1. This makes $a$ a
      $n+1$-dimensional vector. 

      \textbf{Output layer}: $h_\Theta(x) = a^{(3)} = g(z^{(3)}) \text{ where }
      z^{(3)} = \Theta^{(2)}a^{(2)}$ 

      Therefore, for any activation calculation: 
      
      $z^{(j)} = \Theta^{(j-1)}a^{(j-1)}$

      $a^{(j)} = g(z^{(j)})$

      And for any output calculation:

      $z^{(j+1)} = \Theta^{(j)}a^{(j)}$

      $h_\Theta^{(x)} = a^{(j+1)} = g(z^{(j+1)})$
    
    \subsubsection{Multiple output units: one-vs-all}

      $h_\Theta(x) \in \mathbb{R}^4$
      
      If there are 4 output units, first outcome $h_\Theta \approx 
      \begin{bmatrix} 1 \\ 0 \\ 0 \\ 0 \end{bmatrix}$ and so on. 

      So while $y \in\{1,2,3,4\} \text{ previously, now } y^{(i)} 
      \text{ one of } \begin{bmatrix} 1 \\ 0 \\ 0 \\ 0 \end{bmatrix},
      \begin{bmatrix} 0 \\ 1 \\ 0 \\ 0 \end{bmatrix}, \text{ and so on}.$

  \subsection{Learning}

    \subsubsection{Cost Function}

      \begin{itemize}
        \item $L$ = total number of layers (also used to denote output layer)
        \item $s_l$ = number of units (not counting bias) in layer $l$
        \item $K$ = number of units in output layer
      \end{itemize}{
      
      Classification:\\
      Binary = $y = 0 \text{ or } 1, K = 1$\\
      Multi-class = $K \geq 3$

      $h_\Theta(x) \in \mathbb{R}^K$

      $J(\Theta) = -\frac{1}{m} \displaystyle\sum_{i=1}^m \displaystyle
      \sum_{k=1}^K \left[y_k^{(i)} \log(h_\Theta(x^{(i)}))_k + (1-y_k^{(i)})
      \log(1-(h_\Theta(x^{(i)}))_k)\right] + \frac{\lambda}{2m} \displaystyle
      \sum_{l=1}^{L-1} \displaystyle\sum_{i=1}^{s_l} \displaystyle
      \sum_{j=1}^{s_l+1} (\Theta_{ji}^{(l)})^2$

      Nested summation accounts for multiple output nodes.The second summation
      isn't used in a binary classification problem. The regularisation
      accounts for multiple theta matrices.

    \subsubsection{Backpropagation}

      To minimise $J(\Theta)$ a backpropagation algorithm is used. The term
      refers to the fact that the output layer is calculated first, proceeding
      backwards.

      $\delta_j^{(l)}$ = error of activation node $j$ in layer $l, (a_j^{(l)})$

      For the output layer, calculate $\delta_j^{(L)} = y^{(i)} - a_j^{(L)}$

      For the hidden layers, calculate $\delta^{(l)} = (\Theta^{(l)})^T 
      \delta^{(l+1)} .* g'(z^{(l)})$

      $g' \text{ (g prime) } = a^{(l)} .* (1-a^{(l)})$

      \textbf{Procedure}:

      For $i = 1 \text{ to } m$
      \vspace{-1em}
      \begin{itemize}
        \item set $a^{(1)} = x^{(i)}$
        \item perform forward propagation to compute $a^{(l)} \text{ for } 
              l = 2,3,...,L$
        \item Using $y^{(i)}$, compute $\delta{(L)}$
        \item Compute $\delta^{(L-1)},\dots,\delta^{(2)}$
        \item $\Delta_{ij}^{(l)} := \Delta_{ij}^{(l)} + a_j^{(l)}
              \delta_i^{(l+1)}$
      \end{itemize}
      
      Then compute the partial derivative (accounting for bias) gradient
      matrices:

      $D_{ij}^{(l)} := \frac{1}{m} \Delta_{ij}^{(l)} + \lambda
      \Theta_{ij}^{(l)} if j \neq 0$

      $D_{ij}^{(l)} := \frac{1}{m} \Delta_{ij}^{(l)} if = 0$

      \newpage

    \subsubsection{In Practice}
    
      \textbf{Advanced Optimisation:}
      \begin{lstlisting}
      function [jVal, gradient] = costFunction(theta)
      ...
      optTheta = fminunc(@costFunction, initialTheta, options)
      
      % the above works for logistic regression, however with multiple
      % Theta and gradients, they'll need to be unrolled into one vector:

      thetaVec = [Theta1(:); Theta2(:); Theta3(:)];
      DVec = [D1(:); D2(:); D3(:)];

      % to revert, use for e.g. s1=10, s2=10, s3=1:

      Theta1 = reshape(thetaVec(1:110), 10, 11);
      Theta2 = reshape(thetaVec(111:220), 10, 11);
      Theta3 = reshape(thetaVec(221:231), 1, 11);

      % where s is the number of units in a layer and n = s * s+1

      % thetaVec can then be used to calculate the cost function

      function [jval, gradientVec] = costFunction(thetaVec)

      % reshape to get individual Theta
      % use forward/backwards propagation to compute D1,D2,D3 and J(Theta)
      % unroll D1,D2,D3 to get gradientVec
      \end{lstlisting}

      \textbf{Gradient Checking:}

      Gradient checking is used to identify subtle bugs in neural network
      algorithms.

      $\frac{\delta}{\delta\theta} J(\theta) \approx \frac{J(\theta+\epsilon)
      - J(\theta-\epsilon)}{2\epsilon}$

      In octave: 
      
      \texttt{gradApprox = (J(theta+EPSILON) - J(theta-EPSILON)) / (2*EPSILON)}

      With each partial derivative calculation ($\theta_i$), an epsilon is
      added or subtracted to/from $\theta_i$:

      \begin{lstlisting}
      epsilon = 1e-4      % for example - should be small but not too small.
      for i = 1:n,
        thetaPlus = theta;
        thetaPlus(i) = thetaPlus(i) + EPSILON;
        thetaMinus = theta;
        thetaMinus(i) = thetaMinus(i) - EPSILON;
        gradApprox(i) = (J(thetaPlus) - J(thetaMinus)) / (2*EPSILON);
      end;
      \end{lstlisting}

      Then check that \texttt{gradApprox} $\approx$ \texttt{DVec}. Once it does
      gradient checking needs to be turned off as it's computationally
      expensive to run on every iteration of gradient descent or within 
      costFunction.

      \textbf{Random Initialisation:}

      Initialising theta as a matrix of zeros doesn't work in neural networks
      as after each update, parameters for each hidden unit are identical.

      To perform symmetry breaking, use random initialisation:

      $-\epsilon \leq \Theta_{ij}^{(l)} \leq \epsilon$

      \begin{lstlisting}
      Theta1 = rand(10,11) * (2*initEpsilon) - initEpsilon;
      Theat2 = rand(1,11) * (2*initEpsilon) - initEpsilon;
      \end{lstlisting}

    \subsubsection{Implementation}

      \textbf{Network architecture}

      A reasonable default is to use a single hidden layer. If more than one 
      hidden layer is used, the same number of units should be used for each.
      
      Generally speaking, the more hidden units in a hidden layer the better,
      however it will be more computationally expensive. It should be at least
      comparable to the number of input units, however, hidden units can often
      be 2, 3 or 4 times the size of input units. 

      \textbf{Training a Neural Network}
      \vspace{-1em}
      \begin{enumerate}
      \item Randomly initialise weights
      \item Implement forward propagation to get $h_\Theta(x^{(i)})$
      \item Implement code to compute cost function $J(\Theta)$
      \item Implement backward propagation to compute partial derivatives 
            $\frac{\delta}{\delta\Theta_{jk}^{(l)}}J(\Theta)$
      \item Use gradient checking to compare partial derivatives computed
            using backwards propagation against numerical estimate of gradient
            of $J(\Theta)$, then disable gradient checking
      \item Use gradient descent or advanced optimization method with 
            backpropagation to try to minimise $J(\Theta)$ as a function of
            parameters $\Theta$
      \end{enumerate}

      $J(\Theta)$ is non-convex, therefore can get stuck at a local minimum, 
      however this is should not be a problem as it should still get close to
      the global minimum.

  \newpage

\section{Diagnostics and Applications}

  \subsection{Running Diagnostics}

    Debugging:
    \vspace{-1em}
    \begin{itemize}
      \item Get more training examples
      \item Try larger/smaller set of features
      \item Try adding polynomial features
      \item try increasing/decreaing lambda
    \end{itemize}

    \textbf{Evaluate a hypothesis:}

    Place random 30\% of data from the training set into a test set.
    Learn theta from training data (remaining 70\%).

    Linear regression:
    
    $J_{test}(\theta) = (1/(2 * m_{test}))$ * sum of errors squared.
    
    Logistic regression:
    
    $J_{test}(\theta) =$ logistic regression of test set.
    
    Misclassification errror: 
    
    %err($h_\theta(x), y) = 1 \text{ if } h(x) \geq 0.5, y=0\n 
    %\text{ or if } h(x) < 0.5, y=1$\n0 otherwise

    test error = $1/m_{test}$ sum of err($h(x)_{test}, y_{test}$)

    Choosing degree of polynomial: 
    
    Set $d$ = degree of polynomial

    Calculate theta1, 2, ... 10 (i.e. find theta using different degrees of 
    polynomials).

    The problem is $d$ is set to fit the test set, so it might not fit new 
    examples well.

    To address this problem, divide data into 3: training set (60\%), 
    cross validation set (20\%), test set (20\%). 

    After minimising theta for each order, apply theta to the cross validation
     error function.The order with the lowest error is chosen. 

    Problems may be due to bias or variance:

    High bias (underfit): $J_{train}(\theta)$ and $J_{cv}(\theta)$ are high.

    High variance (overfit): $J_{cv}(\theta) \gg J_{train}(\theta)$.
    
    High lambda may underfit data, low lambda may overfit data. 
    
    How to choose regularisation parameter lambda:
    
    Try lambda = 0, 0.01, 0.02, ..., 10 and min J(theta) to theta1, theta2, 
    theta$\_$n respectively. Then apply each theta to CV error function and pick 
    the lowest. Then apply the lowest theta to the test error function. 

    If a learning algorithm is suffering from high bias, more training data will
    not help (i.e. won't fit a straight line any better). Conversely more data
    will help a learning algorithm with high variance. 

    Learning curves will give a better sense of whether there is a bias or 
    variance problem with the learning algorithm. 

    \textbf{Fix high bias}: increase number of features, try more polynomial 
    features,decrease lambda.

    \textbf{Fix high variance}: increase number of training examples, reduce 
    number of features, increase lambda. 

    \textbf{Neural Networks:}

    Small neural network: fewer parameters, more prone to underfitting but 
    computationally cheaper. 

    Large neural network: more parameters, more prone to overfitting and 
    computationally more expensive. Using regularisation can address 
    overfitting, so the larger the number of neurons, the better (potentially).

    Number of hidden layers: To optimise, choose hidden layers=1, 2, 3 and 
    see which performs best in the cross validation error function. 

    Same bias/variance problems as shown in learning curves apply to problems in
    neural networks. 
  
  \subsection{Applications}

    Email spam filter: repeatedly used words in spam emails are the features.
    Within each training example, a word appearing is given 1, not appearing 0.
    y is whether or not the email is spam (or what type of spam it is).

    Recommended approach:
    \vspace{-1em}
    \begin{itemize}
      \item Implement simple algorith quickly and test on cross-validation data.
      \item Plot learning curves to decide on how it can be improved.
      \item Error analysis: manually examine the examples that the algorithm
            made errors on. See whether there is a systematic trend in the type
            of examples it's making errors on. 
    \end{itemize}

    Stemming software (e.g. Porter Stemmer) can be used for natural language
    processing to treat stems of words as the same, such as discount, 
    discounted. It can make mistakes however, such as universe, university.
    Therefore numerical evaluation (CV error) should be used to determine
    the error with and without stemming. 

    \textbf{Skewed classes}: ratio of positive/negative examples is biased.

    Precision = true positives / number of predicted positives (true + false
    +ve)

    Recall = true positives / actual positives (true +ve + false -ve)

    It is possible to trade off precision and recall. If the threshold of
    y=1 is increased (to only predict a postive outcome if very confident)
    then the result will be higher precision, lower recall (avoid false
    +ves). 

    Lower threshold = higher recall, lower precison (avoid false -ves).

    The F score is a forumla to determine the best combination of precision 
    and recall between any number of algorithms.

    F Score = $2\frac{PR}{P+R}$

    There are a number of types of F scores however. Taking the product of 
    P an R means that both values need to be high to attain a high F score.

    \textbf{How much data?}
    
    Useful test: given the input features $x$, can a human expert confidently 
    predict $y$?

    Ask whether a large number of training examples can be obtained. 

  \newpage

\section{Support Vector Machines}

  Optimisation objective function: 
  
  $\min_0 C \displaystyle\sum_{i=1}^m 
  y^{(i)}\text{cost}_1(\theta^Tx^{(i)} + (1-y^{(i)}) \text{cost}_0
  (\theta^Tx^{(i)} + \frac{1}{2} \displaystyle\sum_{(i=1)}^n \theta_j^2$

  The logistic regression optimisation function $A + \lambda B$ is taken and 
  changed to $CA+B$. C is equal to $\frac{1}{\lambda}$.

  Hypothesis $h_\theta(x) = 1 \text{ if } \theta^Tx \geq 0,
  0 \text{ otherwise.}$
    
  SVMs have a larger decision margin than that of linear/logistic regression.
  
  If C is very large, the decision margin is more likely to take outliers into
  account (if data is not linearly separable).
  
  \subsection{Kernels}
  
    Kernels are used to define boundaries of more complex non-linear functions.

    Can think of a decision boundary as theta0 + theta1 * f1 + theta2 * f2 ...
    where f1 = x1.
    
    But as higher order polynomial terms can be more computationally expensive,
    landmarks are used.
    
    fi becomes the similarity of x and l(i). The similarity function is called
    the Guassian Kernel.
    
    $f1 = \exp(- \frac{||x - l^{(1)}||^2}{2\sigma^2})$   

    If $x \approx l(1), f1 \approx 1$.

    If x far from l(1), f1 $\approx$ 0.

    Higher sigma squared means a higher f can be obtained more easily.

    \textbf{Choosing landmarks:}

    $x^{(1)} = l^{(1)}$ 

    $x^{(m)} = l^{(m)}$

    $f_0 = 1\\
    f_1 = similarity(x, l^{(1)})\\
    f_2 = similarity(x, l^{(2)})\\
    f_m = similarity(x, l^{(m)})$

    Hypothesis: given x, compute features $f\in\mathbb{R}^{m+1}$

    Predict y=1 if $(\theta^{(i)})^T f^{(i)} \geq 0$

    Using the SVM optimisation objective function, swap $\theta^Tx^{(i)}$ for 
    $\theta^Tf^{(i)}$.

    \textbf{Choosing SVM Parameters:}
    
    Large C: Lower bias, higher variance.\\
    Small C: Higher bias, lower variance.

    Large sigma squared: features fi vary more smoothly - higher bias, lower
    variance. 

  \subsection{Implementing SVM}

    Use a library such as liblinear and libsvm to solve parameters theta.
    A choice of parameter C and the kernel (similarity function) is still
    needed however. 

    Might use no kernel (a linear kernel) if there are many features but few
    training examples. 

    If using a Guassian kernel, then sigma squared will need to be chosen too.
    This may be used if the number of features is small and the number of 
    examples is large. 

    \begin{lstlisting}
    function f = kernel(x1, x2)       % where x1 is x(i) and x2 is l(j) or x(j)

    % kernel function provided by lib

    return
    \end{lstlisting}

    Important to use feature scaling before using the Guassian kernel. 

    Other kernels include:
    \vspace{-1em}
    \begin{itemize}
      \item Polynomial kernel
      \item String kernel
      \item Chi-square kernel
      \item Histogram intersection kernel
    \end{itemize}

    As with other machine learning algorithms, test on the cross validation
    sample to determine which kernel to use. 

    Multi-class classification: Most SVM packages have multi-class 
    functionality. Otherwise use the one-vs.-all method (train K SVMs).

    Use logistic regression if n $\geq$ m, n=10,000, m=10-1000.

    Use SVM with Guassian kernel if m $>$ n, n=1-1000, m=10-10,000.

    If m $\gg$ n, (n=1-1000, m=50,000+)  create/add more features then use 
    logistic regression or SVM without a kernel.

    A neural network is likely to work well for most of these settings, 
    however it may be slower to train.

  \newpage

\section{Unsupervised Learning}

  For unlabelled data: no $y$.

  Uses: market segmentation, social network analysis, organising computing
  computing clusters, astronomical data anlaysis. 

  \subsection{Clustering}
  
    K-means: n cluster centroids created, where n is number of clusters wanting
    to be created. Each point in the dataset will be evaluated in relation to 
    how close they are to either centroid and are then assigned to that
    centroid. After that, a new set of centroids are created using the newly 
    adjusted means of the clusters. These steps are repeated until k-means
    have converged.
    
    $K$: number of clusters\\
    $k$: index of cluster\\
    $x^{(1)}-x^{(m)}$: training set

    $x^{(i)} \in\mathbb{R}^n$ so $x_0 = 1$ convention is dropped.

    Randomly initialise K cluster centroids $\mu_1,\mu_2,...,\mu_K$

    Repeat for i=1 to m (cluster assignment step), 
    \begin{itemize}
    \vspace{-1em}
      \item $c^{(i)}$ := index(1-K) of cluster centroid closest to $x^{(i)}$ by
            minimising $||x^{(i)} - \mu_k||$. The value of $c^{(i)}$ is the 
            cluster a point has been assigned to.
    \end{itemize}

    Repeat for k=1 to K (move centroid step),
    \begin{itemize}
    \vspace{-1em}
      \item $\mu_k$ := mean of points assigned to cluster k.
    \end{itemize}

    If a centroid hasn't been assigned a point, it can be deleted
    or randomly re-initialised. 

    Optimisation:

    $\mu c^{(i)}$: cluster centroid of cluster to which $x^{(i)}$ has been
    assigned.

    $J(c^{(1)},...,c^{(m)},\mu_1,...,\mu_K) = \frac{1}{m} \sum 
    ||x(i) - muc(i)||^2$

    Cluster assignment: $min J^(c^{(1)},...,c^{(m)},\mu_1,...,\mu_K)$

    The move centroid step is choosing the values of $c^{(i)}$ and $\mu_k$ to
    minimise J.
    
    Initialising K-means:
    
    Random initialisation: should have $K<m$, randomly pick K training
    examples, set\\$\mu_1,...,\mu_K$ equal to these K examples, i.e. $\mu_1 = 
    x^{(i)}, \mu_2 = x^{(j)}$ etc.
    
    K-means can end up at different local optima, so it is run a number of
    times to find the best local optima.

    For i=1 to 100,
    \begin{itemize}
    \vspace{-1em}
      \item Randomly initialise K-means.
      \item Run K-means to get $c^{(1)},...,c^{(m)},\mu_1,...,\mu_K.$
      \item Compute cost function (distortion)  $J(c^{(1)},...,c^{(m)},\mu_1,
            ...,\mu_K)$
    \end{itemize}

    Then pick clustering that gave lowest cost. 

    Choosing K:

    Elbow method: run sequential values of K and plot K against cost function
    J. The `elbow' in the curve should be picked. If the elbow is ambiguous 
    then this method isn't the best. 

    Better method is to evaluate K-means based on a metric for how well it 
    performs for a later/downstream purpose.


  \newpage

\section{Dimensionality Reduction}

  In case two features are the same (e.g. lenth in cm and length in inches, or
  pilot skill and enjoyment which can reduce to pilot aptitude), data
  can be compressed to get only one feature. This speeds up the learning
  algorithm, reduces memory/disk needed to store data and is able to 
  visualise high-dimensional data.  

  $x^{(i)}$ represented by 2 numbers becomes $z^{(i)}$, a real number which 
  refers to the proximity to the projection (PCA is not linear regression). 

  Similarly 3-dimensional data can be reduced to 2-dimensional data. $z_1$ and
  $z_2$ represent a the axis of a 2-dimensional plane within which the data fits.  

  Even 50-dimensional data can be reduced to 10D or even 2D data. There are no 
  ways to visualise more than 3D data however. 
  
  Principle Component Analysis: Finds a surface to project data onto so that it
  can minimise the least squared error. This surface/direction/line is a vector
  $u^{(1)}\in\mathbb{R}^n$. To reduce from n-dimension to k-dimension, find k 
  vectors $u^{(1)}$ to $u^{(k)}$. 

  Preprocessing: perform feature scaling/mean normalisation.

  Compute covariance matrix: $\Sigma = \frac{1}{m} \displaystyle\sum_{i=1}^n 
  (x^{(i)})(x^{(i)})^T$

  Vectorised implementation: \texttt{Sigma = (1/m) * X' * X;}

  Compute eigenvectors of matrix Sigma (singular value decomposition):\\ 
  \texttt{[U,S,V] = svd(Sigma);}

  Sigma = n x n matrix.\\
  U = n x n matrix.

  Take first k parts of U matrix and assign to a Ureduce n x k matrix. The 
  transpose of the Ureduce matrix is the k x n z matrix. 

  \begin{lstlisting}
  Ureduce = U(:,1:k); 
  z = Ureduce' * x;
  \end{lstlisting}

  Reconstruction: To retrieve the original feature vector n, calculate
  Xapprox = Ureduce * z(1).

  Choosing k:

  Typically choose k to be the smallest value so that the avg sq projection
  error / total variation in the data is less than or equal to 0.01, meaning
  that 99\% of variance is retained. 

  $\frac{\frac{1}{m} \sum_{(i=1)}^m ||x^{(i)} - x_{\approx}^{(i)}||^2} 
  {\frac{1}{m} \sum_{i=1}^m ||x^{(i)}||^2}$

  Try PCA with k=1\\Compute Ureduce, $z^{(1)}$ to $z^{(m)}$, $x_{approx}^{(1)}$ 
  to $x_{approx}^{(m)}$\\Check variance. 

  Using \texttt{[U,S,V] = svd(Sigma)} and for a given value of k, the variance
  can be calculated using $1-\frac{\sum_{i=1}^k S_{ii}}{\sum_{i=1}^n S_{ii}}$ 
  i.e. sum Sk / sum Sn.

  Using supervised learning, $(x^{(i)},y^{(i)})$ becomes $(z^{(i)},y^{(i)})$, 
  which can then be used to find $h_\theta^{(z)}$. Mapping $X^{(i)}$ to 
  $z^{(i)}$ should only be done on the training set. That mapping can then be 
  applied to the CV and test sets. 

  PCA should not be used to prevent overfitting, regularisation should be used
  instead. Furthermore, not using PCA should be considered first. If the raw
  data does not achieve the desired result or if the algorithm is particularly
  slow, then consider using PCA. 

  \newpage

\section{Anomaly Detection}

  Density estimation: if $p(x_{test}) < epsilon$, flag as anomaly. 

  Gaussian (normal) distribution: $x~N(\mu,\sigma^2)$, $p(x;\mu,\sigma^2)$.

  $\mu$ = mean
  
  $\sigma^2$ = variance, $\frac{1}{m} * \sum(x - \mu)^2$

  p(x) = sum of product of each p($x^{(i)}$), assuming gaussian distribution and
  a given mu(i)/sigma(i). 

  p(x) = $\pi_{j=i}^n p(x_j;\mu_j, \sigma_j^2)$

  $\Pi$ = sum of products

  Algorithm:

  Choose features xi that might be anomalous

  Fit parameters $\mu_1-\mu_n, \sigma_1^2-\sigma_n^2$. 
  
  $\mu_j = \frac{1}{m} \sum x_j^{(i)}$.
  
  $\sigma^2_j = \frac{1}{m} \sum (x_j^{(i)} - \mu_j)^2$

  Given new example x, compute p(x) using above $\Pi p(x_j)$ algorithm.

  Anomaly if p(x) < epsilon. 

  Assuming labeled data, y=0 if normal, y=1 if anomalous.

  If using a cross validation, data set will be much more skewed towards
  y=0, with a very high accuracy. Instead, evaluating an anomaly detection
  algorithm can be done through true/false positive/negatives, 
  precision/recall or F-score. 
  
  A cross validation set can be used to choose epsilon.  

  Anomaly detection vs. supervised learning:

  AD: very small num positive examples (0-20), large num of negative. If there 
  are many different types of anomalies - hard to learn from positive examples 
  what anomalies look like. E.g. fraud detection, manufacturing, monitoring 
  machines in data center. 
  
  SL: large num of positive/negative examples. Enough positive examples to 
  learn what future positive examples will look like. E.g. spam or health
  classification, weather prediction.

  \begin{lstlisting}
  hist(x); 
  hist(x,50); 
  hist(x.\^0.5,50); 
  % etc to obtain a good Gaussian %distribution
  \end{lstlisting}

  Error analysis for anomaly detection:

  Most common problem: p(x) is comparable for normal and anomalous examples.
  
  Try to create a new feature which x1 can be mapped against. 

  Try to create new features that combine existing features e.g. cpu load
  (optionally squared to capture high cpu load) / network traffic.
  
  This method can potentially capture the different types of anomalies. 
  
  Multivariate Gaussian Distribution: Doesn't model p(x1),p(x2) separately,
  models p(x) all in one go. 

  $p(x) = \frac{1}{(2\pi)^{\frac{n}{2}}|\Sigma|^{\frac{1}{2}}} 
  exp(-\frac{1}{2}(x-\mu)^T\Sigma{-1}(x-\mu))$
  
  mu determines peak of distribution, Sigma the variance. Sigma can also
  determine the correlation of data.

  Parameter fitting: mu = mean, sigma = same as PCA. 

  1. fit model p(x) by setting mu and Sigma.

  2. given a new example x, compute p(x) using Gaussian distribution
  algorithm.

  Then flag anomaly if p(x) $<$ epsilon. 

  When to use original or multivariate models:

  Original: manually create features to capture anomalies where x1,x2 take
  unusual combinations of values. Computationally cheaper, scales better to
  large n). Can be used with small m. 

  Multivariate: automatically caputres correlation between features.
  Computationally more expensive. Must have m $>$ n, or else Sigma is 
  non-invertible. Sigma also non-invertible if features are redundant. 
  
  \newpage
  
\section{Recommender Systems}

  These systems learn which features to use as opposed to features being 
  manually selected. 
  
  $x^{(i)}$ parameters for movie. Contains vector with $x_0=1$ and parameters 
  $x_1-x_m$ (such as how much of an action or romance film it is). 

  $\theta^{(i)}$ parameters for user.
  
  $(\theta^{(i)})^T x^{(i)}$ to determine rating for a movie. 
  
  $r(i,j)$ = 1 if user j has rated movie i (0 otherwise).
  
  $y(i,j)$ = rating by user j for movie i (if defined.
  
  $n_u$ = num users\\
  $n_m$ = num movies\\
  $m^{(j)}$ = num movies rated by user j

  To learn $\theta^{(j)}$ use cost and gradient descent similar to linear 
  regression.

  Collaborative filtering: difficult to determine value of the content
  features. Based on movie ratings, make predictions for parameters $x^{(i)}$. 
  Can also continue to estimate theta and x using previously found values. 
  
  To estimate theta given x, sum over all j (all movies for user). To estimate
  x given theta, sum over all i (all users for movie).

  1. Initialise x, theta to small random values.

  2. Min $J(x^{(1)}-x^{n_m},theta^{1}-theta^{n_m})$ using gradient descent.

  3. For a user with parameters theta and a movie with learned features x, 
  predict a star rating thetaTx

  Y = matrix of (i,j) dimensions.

  Predicted rating (i,j) = $\theta^{(j)})^T x^{(i)}$.

  X = matrix of $(x^{(i)})^T$. Theta = matrix of $(\theta^{(i)})^T$. 

  Multiply to obtain the predicted ratings (named low rank matrix
  factorisation). 

  To find related movies find small $||x^{(i)}-j^{(j)}||$.

  Mean normalisation solves problem of unrated movies. 

  mu vector of average rating for each movie is subtracted from each movie
  rating to calculate Y matrix, used to learn $\theta^{(j)}, x^{(i)}$.
  
  For user j, on movie i predict: $(\theta^{(i)})^T(x^{(i)}) + \mu$. Therefore the predicted
  movie rating for a user is the average rating. 

  \newpage

\section{Large-scale Machine Learning}

  Plot $J_{cv}(\theta)$ against $J_{train}(\theta)$ to see the variance of the 
  data. More data should decrease the variance.

  Can be very expensive to sum over ~1,000,000+ data entries, needed just to 
  compute one step of gradient descent.

  Stochastic gradient descent can be used to calculate gradient descent on
  large datasets. 

  $J_{train}(\theta) = \frac{1}{m} \sum cost(\theta, (x^{(i)}, y^{(i)}))$ where,

  $cost(\theta, (x,y)) = \frac{1}{2} (h(\theta(x^{(i)}) - y^{(i)})^2$

  First randomly shuffle dataset to speed up convergence. The algorithm will 
  change parameters for each training example, rather than changing the
  parameters for every training example. 

  Stochastic gradient descent can also be used for logistic regression and
  neural networks.

  Mini-batch gradient descent: use b examples in each iteration where b is
  the mini-batch size. Like Stochastic GD, no need to iterate over all 
  examples, so it is much faster. b = 2-100 commonly used. 

  Stochastic vs. mini-batch: mini-batch likely to outperform Stochastic only
  if a vectorised implementation is used. The extra parameter b could also
  increase time taken to compute as finding the optimal b is required. 

  To check for convergence in Stochastic GD, plot avg cost every (say) 1000
  iterations (so cost/iterations). Can then change learning rate or number
  of iterations accordingly. Learning rate can be slowly decreased over 
  time using const1/(iterations + const2). This can take time to optimise
  however. Typically not done as keeping alpha constant will converge 
  close enough to the global minimum. 

  Online learning: using a continuous stream of data (e.g. from a continuous 
  stream of users) to train an algorithm. Each training example is discarded
  after use. This method also adapts to changing user preferences. 

  Repeat forever:
  \vspace{-1em}
  \begin{itemize}
  \item Get (x,y) corresponding to user.
  \item Update theta using (x,y).
  \end{itemize}

  Map-reduce approach: divide data so that machines available (each computer)
  have the same ratio of data. Each temp thetaj will be combined afterwards on
  a master server. This is the same as batch gradient descent but is much 
  quicker.
  
  Map-reduce can also be applied on a single computer with mutiple cores.
  
  \newpage
  
\section{Application Case Study}

  Machine Learning pipeline: distinct stages of application development.
  
  Image $\to$ text detection $\to$ character segementation $\to$ character 
  recognition.

  Sliding window detection: for new image, an nxm image patch is taken based on
  the aspect ratio of y=1 and ran through the classifier. After that size patch 
  has run through the entire image, a larger patch is chosen, which is scaled 
  to the original nxm image patch and ran through the classifier, and so on. 
  The step size determines the position of each subsequent image patch. 

  For text detection, text is found individually using the above, with the 
  output an image where white=text, black=no text. Then an expansion 
  operator is used which expands these singular white pixels based on whether
  there exists another white pixel in the near vicinity, so full words/text
  is found. Bounding boxes are then drawn around text with the correct aspect
  ratio (i.e. greater width than height). 

  Character segmentation finds blank space between two characters. y=1 will be
  sliding windows with the split and y=0 will be examples with whole characters
  or entirely blank space. 

  Once each character has been found, it can then be classified using a machine
  learning algorithm. 

  Artificial data synthesis: there are cases where data can be created 
  `artificially'. One example is in text recognition, where synthetic data can
  be created from taking characters from fonts, for a potentially unlimited 
  supply of labelled data for a supervised learning algorithm. 
  
  Similarly distortions can be added to the original training examples to 
  artificially increase the amount of labelled data. This can be applied to
  other problem, such as speech recognition where a voice can be overlayed
  on different background noise. The distortion should be representative of
  the type of distortions that may be seen in the test set. 

  Make sure to have a low bias classifier before synthesising more data, else
  the new examples won't help to optimise the algorithm.

  Consider how long it will take to get 10x as much data, i.e. calculate how
  long it takes to get one example x how much you want. Can also use `crowd
  sourcing' like Amazon Mechanical Turk. 

  Ceiling Analysis: what part of the pipeline should the most time be spent
  improving? 

  Find the accuracy of the overall system. Then manually determine the ground
  truth/classification of each example in the training set for the first part 
  of the pipeline. Find the accuracy of the overall system again but with the 
  100\% accuracy of the first pipeline stagee. Repeat for other parts of the
  pipeline. Using this method it is possible to see which areas of the pipeline
  can be improved the most. 

\end{document}
